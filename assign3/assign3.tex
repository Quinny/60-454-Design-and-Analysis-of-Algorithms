\documentclass[12pt]{article}

\usepackage[margin=1in]{geometry}
\usepackage{amsmath,amsthm,amssymb}
\usepackage[none]{hyphenat}
\usepackage{mathtools}
\usepackage{verbatim}
\usepackage[ruled]{algorithm2e}
\setlength{\parindent}{0pt}

\newcommand{\N}{\mathbb{N}}
\newcommand{\Z}{\mathbb{Z}}
\newcommand{\R}{\mathbb{R}}

\newenvironment{theorem}[2][Theorem]{\begin{trivlist}
\item[\hskip \labelsep {\bfseries #1}\hskip \labelsep {\bfseries #2.}]}{\end{trivlist}}
  \newenvironment{lemma}[2][Lemma]{\begin{trivlist}
\item[\hskip \labelsep {\bfseries #1}\hskip \labelsep {\bfseries #2.}]}{\end{trivlist}}
  \newenvironment{exercise}[2][Exercise]{\begin{trivlist}
\item[\hskip \labelsep {\bfseries #1}\hskip \labelsep {\bfseries #2.}]}{\end{trivlist}}
  \newenvironment{problem}[2][Problem]{\begin{trivlist}
\item[\hskip \labelsep {\bfseries #1}\hskip \labelsep {\bfseries #2.}]}{\end{trivlist}}
  \newenvironment{question}[2][Question]{\begin{trivlist}
\item[\hskip \labelsep {\bfseries #1}\hskip \labelsep {\bfseries #2.}]}{\end{trivlist}}
  \newenvironment{corollary}[2][Corollary]{\begin{trivlist}
\item[\hskip \labelsep {\bfseries #1}\hskip \labelsep {\bfseries #2.}]}{\end{trivlist}}
\DeclarePairedDelimiter{\ceil}{\lceil}{\rceil}
\DeclarePairedDelimiter{\floor}{\lfloor}{\rfloor}
\DeclarePairedDelimiter\abs{\lvert}{\rvert}%

\SetKwInput{KwInput}{Input}
\SetKwInput{KwOutput}{Output}
\SetKwInput{KwLet}{Let}

\begin{document}

\title{Assignment 3}%replace X with the appropriate number
\author{Quinn Perfetto, 104026025\\ %replace with your name
 60-454 Design and Analysis of Algorithms} %if necessary, replace with your course title

\maketitle

\begin{question}{1 (a)}
  \leavevmode \\ \\
  \textbf{Idea:} Sum consecutive elements of the input array until the sum
  exceeds $M$.  Once this happens, add the offending index to the subdivision
  and reset the sum. \\
  \begin{algorithm}[H]
    \caption{Subdivide(W, M)}
    \KwInput{$W[1..n], 0 \leq W[i] \leq M, 1 \leq i \leq n$}
    \KwOutput{$S[1..k]$ such that $S$ is an optimal subdivision of $W$}
    \BlankLine

    \Begin{
      sum := 0\;
      S := $[\ ]$\;
      \For{i $\leftarrow$ 1 \KwTo $n$} {
        sum = sum + $W[i]$\;
        \If{sum $>$ M} {
          append(S, i - 1)\;
          sum := $W[i]$\;
        }
      }
    }
  \end{algorithm}

  \begin{lemma}{1.1} Algorithm Subdivide produces a valid subdivision of the input array $W$
    \leavevmode \\ \\
    We shall show this by inductively proving that after the m$th$ iteration
    of the for loop,
    \begin{align*}
      S \text{ is a valid subdivision of } W[1..m] \land sum = \sum_{j=S_{last} + 1}^{m} W[j]
    \end{align*}

    \textbf{Note:} We take $S_{last}$ to be the last element in $S$ if it exists,
    and 0 otherwise. \\

    \begin{proof}
      (Induction Basis) We first note that sum is initialized to 0.  After
      control reaches line 4 for the first time we have,
      \begin{align*}
        sum = sum + W[1] \Rightarrow sum = W[1] = \sum_{j=1}^{1} W[j]
      \end{align*}
      Note that $S$ was initialized to $[\ ]$.  Since $sum = W[1] \leq M$, control
      will not enter the if statement on line 5, thus $S$ will remain empty and
      $S_{last} = 0$.  Further since $W[1..m=1]$ is a single element list such that
      $W[1] \leq M$, $S = [\ ]$ is vacuously a valid subdivision of $W$. \\

      (Induction Hypothesis) Assume that after $k$ iterations of the for loop,
      \begin{align*}
        S \text{ is a valid subdivision of } W[1..k] \land sum = \sum_{j = S_{last} + 1}^{k} W[j]
      \end{align*}

      (Induction Step)
      \textbf{Case 1:} $sum > M$ \\ \\
      By the induction assumption $S$ is a valid subdivision of $W[1..k]$, by the defintion of
      a valid subdivision we thus have,
      \begin{align*}
        & \sum_{j = S_{last} + 1}^{k} W[j] \leq M & \text{(I)}
      \end{align*}

      After appending $i - 1 = k$ to $S$, $S_{last} = k$.  Therefore (I) is equivalent to,
      \begin{align*}
        \sum_{j = S_{last - 1} + 1}^{S_{last}} W[j] \leq M
      \end{align*}

      Further since $\sum_{j = S_{last} + 1}^{k+1} W[j] = W[k+1] \leq M$ we have
      $S$ is a valid subdivision of \\
      $W[1..k+1]$. \\ \\
      After assigning $sum = W[k + 1]$ we also have $sum = \sum_{S_{last} +1}^{k+1} W[j]$. \\

      \textbf{Case 2:} $sum \leq M$ \\ \\
      Since by our inductive assumption $S$ is a valid subdivision of $W[1..k]$ and,
      \begin{align*}
        sum & = \sum_{j=S_{last} + 1}^{k} W[j] + W[k+1] & \\
            & = \sum_{j=S_{last} + 1}^{k+1} W[j] & \\
            & \leq M &
      \end{align*}
      We have S is a valid subdivision of $W[1..k+1]$.
    \end{proof}

    Therefore by Lemma 1.1, after $n$ iterations $S$ will be a valid parition of
    $W[1..n]$. \\
    Hence the algorithm produces a valid subdivision of $W$.
  \end{lemma}

  \pagebreak
  \begin{lemma}{1.2} Algorithm Subdivide produces an optimal subdivision of the
                     input array in terms of size
    \begin{proof} (Contradiction)
      Suppose to the contrary that Algorithm Subdivide does not produce an optimal
      subdivision of the input array. Let $S$ be the subdivision produced by
      Algorithm Subdivide for some input array $W$, and let $S'$ be a valid
      subdivision of $W$ such that $\abs{S'} < \abs{S}$ (i.e. $S'$ is more
      optimal than $S$).  Since $\abs{S'} < \abs{S}$, $\exists i_{j}, i_{j+1} \in S$
      and $\exists i_{x}, i_{x+1} \in S'$ such that $i_{x} \leq i_{j} < i_{j+1} < i_{x+1}$ (I).
      Such indices must exist for if they didn't the solutions would be equal in size.
      It can be seen that extending any subdivision produced by Algorithm Subdivide
      would result in an invalid subdivision as,
      \begin{align*}
        \exists i_{k}, \sum_{j=i_{k}+1}^{i_{k+1}} W[j] > M
      \end{align*}
      (I) implies that $S'$ contains a subdivision that is an extension of
      a subdivision of $S$, and thus has a sum larger than $M$.  Therefore $S'$
      is an invalid subdivision.  Thus $S$ must be optimal.
    \end{proof}
  \end{lemma}

  \begin{lemma}{1.3} Algorithm Subdivide runs in O(n) time
    \leavevmode \\ \\
    The for loop iterates over each of the $n$ elements of $W$, and performs
    two operations for each iteration (i.e. one addition and one comparison). \\
    We therefore have T(n) = O(2n) = O(n).
  \end{lemma}
\end{question}

\begin{question}{1 (b)}
  The greedy algorithm presented above will not produce an optimal subdivision
  if $W$ contains negative elements.  If $W$ contains negative elements, extending
  a subdivision does not necessarily increase it's sum, and thus greedily ending
  subdivisions does not guarantee optimality. \\

  \textbf{Example:}
  W = $[1, 2, 10, -9]$, $M = 10$ \\
  $S_{greedy}$ = $[2]$, $S_{optimal}$ = $[]$ \\
\end{question}

\begin{question}{2 (a)}
\end{question}

\begin{question}{2 (b)}
  \leavevmode \\ \\
  \textbf{Idea:} Let $D[i, j]$ represent the least difference mapping between
  $H[i..n]$ and $S[j..m]$.  For any $i,\ j$ two options arise:
  \begin{itemize}
    \item Pair $H_{i}$ with $S_{j}$, in which case $D[i,\ j] = \abs{H_{i} - S_{j}} + D[i + 1, j + 1]$
    \item Don't pair $H_{i}$ with $S_{j}$, in which case $D[i,\ j] = D[i,\ j + 1]$
  \end{itemize}
  The optimal result is thus the minimum of the two options.
  
  \pagebreak

  \textbf{Base cases:}
  \begin{itemize}
    \item $D[n, m]$ = the least difference mapping between $H[n..n]$ and $S[m..m]$ = $\abs{H_{n} - S_{m}}$
    \item $D[i,\ m] = \infty$, $1 \leq i \leq n - 1$ since each value in $H$ must map to a distinct value in $S$
    \item $D[n + 1, i] = 0$, $1 \leq i \leq m$ since $H[n+1..n]$ is empty
  \end{itemize}

  \begin{algorithm}[H]
    \caption{LeastDifferenceMapping(H, S)}
    \KwInput{$ H=\{h_{j}\ |\ 1 \leq j \leq n\}, S = \{S_{j}\ |\ 1 \leq j \leq m\},\ n \leq m$}
    \KwOutput{$min(\sum_{i=0}^{n}\abs{H[i] - S[i]})$}
    \BlankLine

    \Begin{
      \For{i $\leftarrow$ 1 \KwTo $m$} {
        $D[n + 1,\ i]$ = 0\;
      }
      \For{i $\leftarrow$ 1 \KwTo $n - 1$} {
        $D[i,\ m]$ = $\infty$\;
      }

      \BlankLine
      SortAscending(H)\;
      SortAscending(S)\;
      $D[n, m] = \abs{H[n] - S[m]}$ \;

      \For{i $\leftarrow$ n \KwTo 1} {
        \For{j $\leftarrow$ m - 1 \KwTo 1} {
          $D[i, j]$ = min($\abs{H[i] - S[j]} + D[i+1,\ j+1],\ D[i,\ j+1]$)\;
        }
      }
      \KwRet{$D[1, 1]$}
    }
  \end{algorithm}

  \begin{lemma}{2.1} Algorithm LeastDifferenceMapping correctly produces the mapping from $H$ to $S$ such that $\sum_{j=1}^{n} \abs{h_{j} - s{j}}$ is minimized
    \leavevmode \\ \\
    (The optimal substructure) \\
    Consider a one-to-one mapping from two sequences sorted in ascending order\\
    $h_{i}h_{i+1}...h_{n}$ to $s_{j}s_{j+1}...s_{m}$. \\ \\
    We define the least difference mapping as a mapping that minimizes $\sum_{x=i}^{n} \abs{h_{x} - s_{x}}$. \\ \\
    Let $D[i,\ j] = $ the summation of the differences in the least difference mapping of $h_{i}h_{i+1}...h_{n}$ to $s_{j}s_{j+1}...s_{m}$. \\ \\
    In any least difference mapping $h_{i}h_{i+1}...h_{n}$ to $s_{j}s_{j+1}...s_{m}$,
    \begin{enumerate}
      \item If $h_{i}$ is mapped to $s_{j}$, then $h_{i+1}...h_{n}$ is mapped to
        $s_{j+1}...s_{m}$ and must be a least difference mapping.  Otherwise,
        a more optimal mapping from $h_{i+1}...h_{n}$ to $s_{j+1}...s_{m}$ combined
        with the mapping from $h_{i}$ to $s_{j}$ would produce a mapping with a
        smaller difference, a contradiction! \\ \\
        It follows that $D[i, j] = \abs{h_{i} - s_{j}} + D[i + 1,\ j + 1]$.

      \item If $h_{i}$ is not mapped to $s_{j}$, then $h_{i}...h_{n}$ is mapped
        to $s_{j+1}...s_{m}$ and must must be a least difference mapping.  Since
        the sequences are sorted, by the property proven in 2 (a) any other
        mapping could be swapped to decrease the sum, thus it must be a minimum. \\ \\
        It follows that $D[i, j] = D[i, j + 1]$.
    \end{enumerate}

    We thus obtain the following recurrence:
    \begin{align*}
      D[i, j] = min\{\abs{h_{i} - s_{j}} + & D[i + 1, j + 1], & \\
                                           & D[i, j + 1]\}
    \end{align*}

    Clearly,
    \begin{align*}
      & D[n, m] = \abs{h_{n} - s_{m}} & \text{(Only one possible mapping)} \\
      & D[n + 1, i] = 0 & \text{(H is empty)} \\
      & D[i, m] = \infty & \text{(Not enough elements in S for a distinct mapping)}
    \end{align*}
  \end{lemma}

  \begin{lemma}{2.2} Algorithm LeastDifferenceMapping runs in O($mlgm$ + mn) time
    \leavevmode \\ \\
    \textbf{Initialization:} Initializing the first and second base cases perform
    $m$ and $n - 1$ operations respectively. \\

    \textbf{Sorting:} Sorting $H$ and $S$ requires at most $nlgn$ and $mlgm$ operations
    respectively. \\

    \textbf{Filling the table:} The outer loop performs $n$ iterations, while the
    inner loop performs $m-1$ iterations.  Each of the $m-1$ iterations performs
    a constant amount of work, we therefore have $n(m-1) = nm - n = O(mn)$ total operations. \\

    \textbf{Total:} In summation,
    \begin{align*}
      T(n, m) & = O(m) + O(n - 1) + O(nlgn) + O(mlgm) + O(mn) & \\
           & = O(mlgm) + O(mn) & \text{($n \leq m$)} \\
           & = O(mlgm + mn) & \\
    \end{align*}
  \end{lemma}
\end{question}

\begin{question}{3 (a)}
  \leavevmode
  \begin{proof}
    Let G = (V, E) be a connected simple graph such that $\exists j, 1 \leq j \leq d_{1} + 1,\ \{v_{1}, v_{j}\} \notin E$. \\
    Let $v_{j} \in V$, $1\leq j \leq d_{1} + 1,\ \{v_{1}, v_{j}\} \notin E$. \\
  \end{proof}
\end{question}

\end{document}
