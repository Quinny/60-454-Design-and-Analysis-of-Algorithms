\documentclass[12pt]{article}

\usepackage[margin=1in]{geometry}
\usepackage{amsmath,amsthm,amssymb}
\usepackage[none]{hyphenat}
\usepackage{mathtools}
\usepackage{verbatim}
\usepackage[ruled]{algorithm2e}
\setlength{\parindent}{0pt}

\newcommand{\N}{\mathbb{N}}
\newcommand{\Z}{\mathbb{Z}}
\newcommand{\R}{\mathbb{R}}

\newenvironment{theorem}[2][Theorem]{\begin{trivlist}
\item[\hskip \labelsep {\bfseries #1}\hskip \labelsep {\bfseries #2.}]}{\end{trivlist}}
  \newenvironment{lemma}[2][Lemma]{\begin{trivlist}
\item[\hskip \labelsep {\bfseries #1}\hskip \labelsep {\bfseries #2.}]}{\end{trivlist}}
  \newenvironment{exercise}[2][Exercise]{\begin{trivlist}
\item[\hskip \labelsep {\bfseries #1}\hskip \labelsep {\bfseries #2.}]}{\end{trivlist}}
  \newenvironment{problem}[2][Problem]{\begin{trivlist}
\item[\hskip \labelsep {\bfseries #1}\hskip \labelsep {\bfseries #2.}]}{\end{trivlist}}
  \newenvironment{question}[2][Question]{\begin{trivlist}
\item[\hskip \labelsep {\bfseries #1}\hskip \labelsep {\bfseries #2.}]}{\end{trivlist}}
  \newenvironment{corollary}[2][Corollary]{\begin{trivlist}
\item[\hskip \labelsep {\bfseries #1}\hskip \labelsep {\bfseries #2.}]}{\end{trivlist}}
\DeclarePairedDelimiter{\ceil}{\lceil}{\rceil}
\DeclarePairedDelimiter{\floor}{\lfloor}{\rfloor}
\DeclarePairedDelimiter\abs{\lvert}{\rvert}%

\SetKwInput{KwInput}{Input}
\SetKwInput{KwOutput}{Output}
\SetKwInput{KwLet}{Let}
\SetKwProg{Fn}{Function}{}{}

\begin{document}

\title{Assignment 3}%replace X with the appropriate number
\author{Quinn Perfetto, 104026025\\ %replace with your name
 60-454 Design and Analysis of Algorithms} %if necessary, replace with your course title

\maketitle

\begin{question}{1 (a)}
  \leavevmode \\ \\
  \textbf{Idea:} Sum consecutive elements of the input array until the sum
  exceeds $M$.  Once this happens, add the offending index to the subdivision
  and reset the sum. \\
  \begin{algorithm}[H]
    \caption{Subdivide(W, M)}
    \KwInput{$W[1..n], 0 \leq W[i] \leq M, 1 \leq i \leq n$}
    \KwOutput{$S[1..k]$ such that $S$ is an optimal subdivision of $W$}
    \BlankLine

    \Begin{
      sum := 0\;
      S := $[\ ]$\;
      \For{i $\leftarrow$ 1 \KwTo $n$} {
        sum = sum + $W[i]$\;
        \If{sum $>$ M} {
          append(S, i - 1)\;
          sum := $W[i]$\;
        }
      }
    }
  \end{algorithm}

  \begin{lemma}{1.1} Algorithm Subdivide produces a valid subdivision of the input array $W$
    \leavevmode \\ \\
    We shall show this by inductively proving that after the m$th$ iteration
    of the for loop,
    \begin{align*}
      S \text{ is a valid subdivision of } W[1..m] \land sum = \sum_{j=S_{last} + 1}^{m} W[j]
    \end{align*}

    \textbf{Note:} We take $S_{last}$ to be the last element in $S$ if it exists,
    and 0 otherwise. \\

    \begin{proof}
      (Induction Basis) We first note that sum is initialized to 0.  After
      control reaches line 4 for the first time we have,
      \begin{align*}
        sum = sum + W[1] \Rightarrow sum = W[1] = \sum_{j=1}^{1} W[j]
      \end{align*}
      Note that $S$ was initialized to $[\ ]$.  Since $sum = W[1] \leq M$, control
      will not enter the if statement on line 5, thus $S$ will remain empty and
      $S_{last} = 0$.  Further since $W[1..m=1]$ is a single element list such that
      $W[1] \leq M$, $S = [\ ]$ is vacuously a valid subdivision of $W[1..m]$. \\

      (Induction Hypothesis) Assume that after $k$ iterations of the for loop,
      \begin{align*}
        S \text{ is a valid subdivision of } W[1..k] \land sum = \sum_{j = S_{last} + 1}^{k} W[j]
      \end{align*}

      (Induction Step)
      \textbf{Case 1:} $sum > M$ \\ \\
      By the induction assumption $S$ is a valid subdivision of $W[1..k]$.  By the defintion of
      a valid subdivision we thus have,
      \begin{align*}
        & \sum_{j = S_{last} + 1}^{k} W[j] \leq M & \text{(I)}
      \end{align*}

      After appending $i - 1 = k$ to $S$, $S_{last} = k$.  Therefore (I) is equivalent to,
      \begin{align*}
        \sum_{j = S_{last - 1} + 1}^{S_{last}} W[j] \leq M
      \end{align*}

      Further since $\sum_{j = S_{last} + 1}^{k+1} W[j] = W[k+1] \leq M$ we have
      $S$ is a valid subdivision of \\
      $W[1..k+1]$. \\ \\
      After assigning $sum = W[k + 1]$ we also have $sum = \sum_{S_{last} +1}^{k+1} W[j]$. \\

      \textbf{Case 2:} $sum \leq M$ \\ \\
      Since by our inductive assumption $S$ is a valid subdivision of $W[1..k]$ and,
      \begin{align*}
        sum & = \sum_{j=S_{last} + 1}^{k} W[j] + W[k+1] & \\
            & = \sum_{j=S_{last} + 1}^{k+1} W[j] & \\
            & \leq M &
      \end{align*}
      We have S is a valid subdivision of $W[1..k+1]$.
    \end{proof}

    Therefore by Lemma 1.1, after $n$ iterations $S$ will be a valid subdivision of
    $W[1..n]$. \\
    Hence the algorithm produces a valid subdivision of $W$.
  \end{lemma}

  \pagebreak
  \begin{lemma}{1.2} Algorithm Subdivide produces an optimal subdivision of the
                     input array in terms of size
    \begin{proof} (Contradiction)
      Suppose to the contrary that Algorithm Subdivide does not produce an optimal
      subdivision of the input array. Let $S$ be the subdivision produced by
      Algorithm Subdivide for some input array $W$, and let $S'$ be a valid
      subdivision of $W$ such that $\abs{S'} < \abs{S}$ (i.e. $S'$ is more
      optimal than $S$).  Since $\abs{S'} < \abs{S}$, $\exists i_{j}, i_{j+1} \in S$
      and $\exists i_{x}, i_{x+1} \in S'$ such that $i_{x} \leq i_{j} < i_{j+1} < i_{x+1}$ (I).
      Such indices must exist for if they didn't the solutions would be equal in size.
      Since $W[i] > 0,\ 1 \leq i \leq n$, it can be seen that extending any subdivision
      produced by Algorithm Subdivide would result in an invalid subdivision as,
      \begin{align*}
        \exists i_{k}, \sum_{j=i_{k}+1}^{i_{k+1}} W[j] > M
      \end{align*}
      (I) implies that $S'$ contains a subdivision that is an extension of
      a subdivision of $S$, and thus must have a sum larger than $M$.  Therefore $S'$
      is an invalid subdivision.  Hence $S$ must be optimal.
    \end{proof}
  \end{lemma}

  \begin{lemma}{1.3} Algorithm Subdivide runs in O(n) time
    \leavevmode \\ \\
    The for loop iterates over each of the $n$ elements of $W$, and performs
    two operations for each iteration (i.e. one addition and one comparison). \\
    We therefore have T(n) = O(2n) = O(n).
  \end{lemma}
\end{question}

\begin{question}{1 (b)}
  The greedy algorithm presented above will not produce an optimal subdivision
  if $W$ contains negative elements.  If $W$ contains negative elements, extending
  a subdivision does not necessarily increase it's sum, and thus greedily ending
  subdivisions does not guarantee optimality. \\

  \textbf{Example:}
  W = $[1, 2, 10, -9]$, $M = 10$ \\
  $S_{greedy}$ = $[2]$, $S_{optimal}$ = $[]$ \\
\end{question}

\begin{question}{2 (a)}
  \begin{proof}
    Given $s_{i} < s_{j} \land h_{i} < h_{j}$, by symmetry 3 different cases arise: \\ \\
    \textbf{Case 1:} $s_{i} < s_{j} \leq h_{i} < h_{j}$
    We thus have,
    \begin{align*}
      \abs{h_{j} - s_{j}} + \abs{h_{i} - s_{i}} = h_{j} - s_{j} + h_{i} - s_{i}
    \end{align*}
    And,
    \begin{align*}
      \abs{h_{j} - s_{i}} + \abs{h_{i} - s_{j}} = h_{j} - s_{i} + h_{i} - s_{j}
    \end{align*}
    I.e.
    \begin{align*}
      h_{j} - s_{j} + h_{i} - s_{i} = h_{j} - s_{i} + h_{i} - s_{j} & \Rightarrow \abs{h_{j} - s_{i}} + \abs{h_{i} - s_{j}} = \abs{h_{j} - s_{i}} + \abs{h_{i} - s_{j}} & \\
                                                                    & \Rightarrow \abs{h_{j} - s_{i}} + \abs{h_{i} - s_{j}} \leq \abs{h_{j} - s_{i}} + \abs{h_{i} - s_{j}} &
    \end{align*}

    \textbf{Case 2:} $s_{i} \leq h_{i} \leq s_{j} \leq h_{j}$
    We thus have,
    \begin{align*}
      \abs{h_{j} - s_{j}} + \abs{h_{i} - s_{i}} = h_{j} - s_{j} + h_{i} - s_{i}
    \end{align*}
    And,
    \begin{align*}
      \abs{h_{j} - s_{i}} + \abs{h_{i} - s_{j}} = h_{j} - s_{i} + s_{j} - h_{i}
    \end{align*}
    I.e.,
    \begin{align*}
      s_{j} \geq h_{i} & \Rightarrow s_{j} - h_{i} \geq 0 & \\
                       & \Rightarrow -(s_{j} - h_{i}) \leq s_{j} - h_{i} & \\
                       & \Rightarrow h_{j} - s_{j} - s_{j} - h_{i} \leq h_{j} - s_{i} + s_{j} - h_{i} &\\
                       & \Rightarrow \abs{h_{j} - s_{j}} + \abs{h_{i} - s_{i}} \leq \abs{h_{j} - s_{i}} + \abs{h_{i} - s_{j}} & \\
    \end{align*}

    \textbf{Case 3:} $s_{i} \leq h_{i} < h_{j} \leq s_{j}$
    We thus have,
    \begin{align*}
      \abs{h_{j} - s_{j}} + \abs{h_{i} - s_{i}} = s_{j} - h_{j} + h_{i} - s_{i}
    \end{align*}
    And,
    \begin{align*}
      \abs{h_{j} - s_{i}} + \abs{h_{i} - s_{j}} = h_{j} - s_{i} + s_{j} - h_{i}
    \end{align*}
    I.e.,
    \begin{align*}
      h_{j} \geq h_{i} & \Rightarrow h_{j} - h_{i} \geq 0 & \\
                       & \Rightarrow -(h_{j} - h_{i}) \leq h_{j} - h_{i} & \\
                       & \Rightarrow s_{j} - h_{j} + h_{i} - s_{i} \leq h_{j} - s_{i} + s_{j} - h_{i} &\\
                       & \Rightarrow \abs{h_{j} - s_{j}} + \abs{h_{i} - s_{i}} \leq \abs{h_{j} - s_{i}} + \abs{h_{i} - s_{j}} & \\
    \end{align*}

    The remaining cases can be expressed by swapping $h_{i}$ with
    $s_{i}$ and $s_{j}$ with $h_{j}$ and are thus omitted.\\ 
    Therefore the statement holds true for all cases.\\
    Hence $s_{i} < s_{j} \land h_{i} < h_{j} \Rightarrow \abs{h_{j} - s_{j}} + \abs{h_{i} - s_{i}} \leq \abs{h_{j} - s_{i}} + \abs{h_{i} - s_{j}}$ 
  \end{proof}
\end{question}

\begin{question}{2 (b)}
  \leavevmode \\ \\
  \textbf{Idea:} Let $D[i, j]$ represent the least difference mapping between
  $H[i..n]$ and $S[j..m]$.  For any $i,\ j$ two options arise:
  \begin{itemize}
    \item Pair $H_{i}$ with $S_{j}$, in which case $D[i,\ j] = \abs{H_{i} - S_{j}} + D[i + 1, j + 1]$
    \item Don't pair $H_{i}$ with $S_{j}$, in which case $D[i,\ j] = D[i,\ j + 1]$
  \end{itemize}
  The optimal result is thus the minimum of the two options. \\

  \textbf{Base cases:}
  \begin{itemize}
    \item $D[n, m]$ = the least difference mapping between $H[n..n]$ and $S[m..m]$ = $\abs{H_{n} - S_{m}}$
    \item $D[i,\ m] = \infty$, $1 \leq i \leq n - 1$ since each value in $H$ must map to a distinct value in $S$
    \item $D[n + 1, i] = 0$, $1 \leq i \leq m$ since $H[n+1..n]$ is empty
  \end{itemize}

  \begin{algorithm}[H]
    \caption{LeastDifferenceMapping(H, S)}
    \KwInput{$ H=\{h_{j}\ |\ 1 \leq j \leq n\}, S = \{S_{j}\ |\ 1 \leq j \leq m\},\ n \leq m$}
    \KwOutput{$min(\sum_{i=0}^{n}\abs{H[i] - S[i]})$}
    \BlankLine

    \Begin{
      \For{i $\leftarrow$ 1 \KwTo $m$} {
        $D[n + 1,\ i]$ = 0\;
      }
      \For{i $\leftarrow$ 1 \KwTo $n - 1$} {
        $D[i,\ m]$ = $\infty$\;
      }

      \BlankLine
      SortAscending(H)\;
      SortAscending(S)\;
      $D[n, m] = \abs{H[n] - S[m]}$ \;

      \For{i $\leftarrow$ n \KwTo 1} {
        \For{j $\leftarrow$ m - 1 \KwTo 1} {
          $D[i, j]$ = min($\abs{H[i] - S[j]} + D[i+1,\ j+1],\ D[i,\ j+1]$)\;
        }
      }
      \KwRet{$D[1, 1]$}
    }
  \end{algorithm}

  \begin{lemma}{2.1} Algorithm LeastDifferenceMapping correctly produces the mapping from $H$ to $S$ such that $\sum_{j=1}^{n} \abs{h_{j} - s{j}}$ is minimized
    \leavevmode \\ \\
    (The optimal substructure) \\
    Consider a one-to-one mapping from two sequences sorted in ascending order\\
    $h_{i}h_{i+1}...h_{n}$ to $s_{j}s_{j+1}...s_{m}$. \\ \\
    We define the least difference mapping as a mapping that minimizes $\sum_{x=i}^{n} \abs{h_{x} - s_{x}}$. \\ \\
    Let $D[i,\ j] = $ the summation of the differences in the least difference mapping of
    $h_{i}h_{i+1}...h_{n}$ to $s_{j}s_{j+1}...s_{m}$. \\ \\
    In any least difference mapping $h_{i}h_{i+1}...h_{n}$ to $s_{j}s_{j+1}...s_{m}$,
    \begin{enumerate}
      \item If $h_{i}$ is mapped to $s_{j}$, then $h_{i+1}...h_{n}$ is mapped to
        $s_{j+1}...s_{m}$ and must be a least difference mapping.  Otherwise,
        a more optimal mapping from $h_{i+1}...h_{n}$ to $s_{j+1}...s_{m}$ combined
        with the mapping from $h_{i}$ to $s_{j}$ would produce a mapping with a
        smaller difference, a contradiction! \\ \\
        It follows that $D[i, j] = \abs{h_{i} - s_{j}} + D[i + 1,\ j + 1]$.

      \item If $h_{i}$ is not mapped to $s_{j}$, then $h_{i}...h_{n}$ is mapped
        to $s_{j+1}...s_{m}$ and must must be a least difference mapping.  Since
        the sequences are sorted, the property proven in 2 (a) shows that any other
        mapping could be swapped to decrease the sum, thus it must be a minimum. \\ \\
        It follows that $D[i, j] = D[i, j + 1]$.
    \end{enumerate}

    We thus obtain the following recurrence:
    \begin{align*}
      D[i, j] = min\{\abs{h_{i} - s_{j}} + & D[i + 1, j + 1], & \\
                                           & D[i, j + 1]\}
    \end{align*}

    Clearly,
    \begin{align*}
      & D[n, m] = \abs{h_{n} - s_{m}} & \text{(Only one possible mapping)} \\
      & D[n + 1, i] = 0 & \text{(H is empty)} \\
      & D[i, m] = \infty & \text{(Not enough elements in S for a distinct mapping)}
    \end{align*}
  \end{lemma}

  \begin{lemma}{2.2} Algorithm LeastDifferenceMapping runs in O($mlgm$ + mn) time
    \leavevmode \\ \\
    \textbf{Initialization:} Initializing the first and second base cases perform
    $m$ and $n - 1$ operations respectively. \\

    \textbf{Sorting:} Sorting $H$ and $S$ requires at most $nlgn$ and $mlgm$ operations
    respectively. \\

    \textbf{Filling the table:} The outer loop performs $n$ iterations, while the
    inner loop performs $m-1$ iterations.  Each of the $m-1$ iterations performs
    a constant amount of work, we therefore have $n(m-1) = nm - n = O(mn)$ total operations. \\

    \textbf{Total:} In summation,
    \begin{align*}
      T(n, m) & = O(m) + O(n - 1) + O(nlgn) + O(mlgm) + O(mn) & \\
           & = O(mlgm) + O(mn) & \text{($n \leq m$)} \\
           & = O(mlgm + mn) & \\
    \end{align*}
  \end{lemma}
\end{question}

\pagebreak

\begin{question}{3 (a)}
  \leavevmode
  \begin{proof}
    Let G = (V, E) be a connected simple graph such that $\exists j, 1 \leq j \leq d_{1} + 1,\ \{v_{1}, v_{j}\} \notin E$. \\ \\
    We note that since G is a simple graph $\{v_{1}, v_{1}\} \notin E$, therefore the above statement can be re-expressed
    as $\exists j,\ 2 \leq j \leq d_{1} + 1,\ \{v_{1}, v_{j}\} \notin E$. \\ \\
    Let $v_{j} \in V, \{v_{1}, v_{j}\} \notin E, 2 \leq j \leq d_{1} + 1$. \\
    Since $v_{1}$ is adjacent to $d_{1}$ verticies, and $\abs{\{v_{i}\ |\ 2 \leq i \leq d_{1} + 1\} - \{v_{j}\}} = d_{1} - 1$,
    $v_{1}$ must be adjacent to a vertex outside of this range.  That is,
    \begin{align*}
      \exists v_{\ell} \in V,\ d_{1} + 1 < \ell \leq n, \{v_{1}, v_{\ell}\} \in E
    \end{align*}

    By transitivity $j < \ell$, therefore $d_{j} \geq d_{\ell}$.  Further since
    $v_{1}$ is adjacent to $v_{\ell}$ and not adjacent to $v_{j}$,
    there must exist some vertex, namely $u (\neq v_{j}, v_{\ell})$,
    that is adjacent to $v_{j}$ and not $v_{\ell}$. Hence we have,
    \begin{align*}
      \{v_{1}, v_{j}\} \notin E \land \{u, v_{\ell}\} \notin E \land
      \{v_{1}, v_{\ell}\} \in E \land \{u, v_{j}\} \in E
    \end{align*}

    Such that,
    \begin{align*}
      2 \leq j \leq d_{1}+1 \land 1 \leq j \leq d_{1} + 1 < \ell \leq n \land u \in V - \{v_{\ell}, v_{j}\}
    \end{align*}
  \end{proof}
\end{question}

\begin{question}{3 (b)}
  \leavevmode
  \begin{proof}
    We must construct a graph $G'$ with the same degree sequence as $G$, but
    $\{v_{1}, v_{j}\} \in E'$. \\

    $\forall v_{j} \in V,\ \{v_{1}, v_{j}\} \notin E,\ 2 \leq j \leq d_{1} + 1$
    we assume the existance of the corresponding $v_{\ell}$, and $u$ in $G$ by the previously
    proven theorem in Question 3 a. \\

    $E'$ can then be constructed by performing the following transformations to $E$:
    \begin{itemize}
      \item Connect $v_{1}$ to $v_{j}$
      \item Connect $u$ to $v_{\ell}$
      \item Disconnect $v_{1}$ from $v_{\ell}$ (Note that this restores each
        to their original degrees)
      \item Disconnect $v_{j}$ from $u$ (Note that this restores each to their
        original degrees)
    \end{itemize}
    Thus $G'$ contains the same degree sequence as $G$, but contains an edge
    from $v_{1}$ to $v_{j}$.
  \end{proof}
\end{question}

\begin{question}{3 (c)}
  \leavevmode \\ \\
  \textbf{Idea:} After sorting the degree sequence, we can apply the theorem
  proved in Question 3 (b) to assert that there exists a graph with an identical
  degree sequence such that \\
  $\forall v_{i}, \{v_{1}, v_{i}\} \in E, 2 \leq i \leq d_{1}+1$.  Thus by
  removing $v_{1}$ from the graph, we reduce the degree of the following $d_{1}$
  vertices by $1$.  This process can be continued until:
  \begin{enumerate}
      \item We are left with a null graph.  I.e. decomposing the graph one vertex
        at a time leads to an empty graph.
      \item $d_{1} > \abs{V} - 1$ or $d_{1} < 0$.  In this case the graph is
        invalid, as the degree of a single vertex in a simple graph must be
        $\in [0, n-1]$.
  \end{enumerate}
  The degree sequence shall be stored in a \textit{linked list} to allow for
  constant time removal and reordering of the elements.

  \begin{algorithm}[H]
    \caption{IsValidGraph(D)}
    \KwInput{$D[1..n]$ a \textit{linked list} of integers}
    \KwOutput{Whether or not a graph exists with the degree sequence D}
    \BlankLine


    \Begin{
      SortDescending(D)\;
      \KwRet{IsValidGraph'(D)}\;
    }

    \BlankLine
    \Fn{IsValidGraph'($D_{sorted}$)}{
      \Begin{
        \If{$D_{sorted}$ is empty} {
          \KwRet{true}\;
        }
        \BlankLine

        \tcp{Remove the first element and return it}
        $d_{1}$ = pop\_first($D_{sorted}$)\;
        \If {$d_{1} > \abs{D_{sorted}}$ or $d_{1} < 0$} {
          \KwRet{false}\;
        }

        \BlankLine
        \tcp{Pointer to the head of the linked list}
        ptr := head($D_{sorted}$)\;
        \For{i $\leftarrow$ 1 \KwTo $d_{1}$} {
          ptr.value = ptr.value - 1\;
          ptr = ptr.next\;
        }
        \BlankLine
        MaintainSort($D_{sorted}$)\;
        \KwRet{IsValidGraph'($D_{sorted}$)}\;
      }
    }
  \end{algorithm}

  \begin{lemma}{3.1} Algorithm IsValidGraph correctly determines if a graph exists
    with the provided degree sequence
    \leavevmode \\ \\
    \textbf{Note:} IsValidGraph simply sorts the degree sequence in descending order
    and then calls IsValidGraph'.  The remainder of this proof will reference
    IsValidGraph', and assume the initial input was sorted.  This sorting will
    be accounted for in the following time complexity analysis.\\

    We shall show correctness by induction on the input size $n$. \\

    (Induction Basis) $n = 0$ The null graph is a graph with an empty degree sequence.
    Therefore there does exist a graph when $n = 0$.  The algorithm will clearly
    return true as $D_{sorted}$ will be empty upon the initial call.  Therefore
    the algorithm works correctly. \\

    (Induction Hypothesis) Suppose $\forall k < n\ (n > 0)$, the algorithm correctly
    determines if there exists a graph with the given degree sequence of size $k$. \\

    (Induction Step) Let $D'$ be a degree sequence of size $n$ sorted in descending order.
    By the theorem proven in Question 3 (b), there must exist a graph with the same
    degree sequence such that
    $\forall v_{i}, \{v_{1}, v{i}\} \in E,\ 2 \leq i \leq d_{1} + 1$.
    Following the call to pop\_first on Line 4, $d_{1}$ contains the maximum value
    in $D'$ (since it is assumed to be sorted in descending order), and
    $\abs{D'} = n - 1$.\\

    If $d_{1} > |D'| = n - 1$ we have an invalid degree sequence,
    as the maximum degree in any simple graph is $n - 1$. Additionally if
    $d_{1} < 0$ the degree sequence is invalid as no vertex can have a negative
    degree.  The algorithm correctly reports this and returns false.\\

    Since $\{v_{1}, v_{i}\} \in E, 2 \leq i \leq d_{1}+1$,
    removing $d_{1}$ from the degree sequence will result in a new graph with
    degree sequence $d_{2} - 1, d_{3} - 1,..,d_{d_{1}+1} - 1,d_{d_{1}},..,d_{n}$. \\

    The following for loop will decrement each of
    $d_{2},..,d_{d_{1}+1}$ by 1 from $D'$ producing\\
    $d_{2}-1,d_{3}-1,..,d_{d_{1}+1}-1,d_{d_{1}},..,d_{n}$ i.e. $D'$ will represent
    the degree sequence of $V - \{v_{1}\}$. \\

    MaintainSort has previously been proven to correctly resort the elements
    of a linked list, thus $D'$ will be sorted in descending order following
    the call to it. \\

    Finally, IsValidGraph'($D'$) will correctly identify whether a graph
    exists with the degree sequence $D'$ by the induction hypothesis since
    $|D'| = n - 1 < n$ and $D'$ is sorted in descending order. \\

    Therefore $\forall n > 0$ Algorithm IsValidGraph correctly determines if a
    graph exists with a given degree sequence of size $n$. Hence, the correctness
    is proved.
  \end{lemma}

  \begin{lemma}{3.2} Algorithm IsValidGraph graphs runs in O($nlgn$ + D) time where
    $n = \abs{V}$ and $D = \sum_{i=1}^{n} d_{i}$
    \leavevmode \\

    \textbf{Sorting:} Sorting the degree sequence requires at most $nlgn$ operations \\

    \textbf{Recursive calls:} Since $IsValidGraph'$ removes at most 1 element from
    $V$ and terminates when $V$ is empty, we have at most $n$ calls to IsValidGraph'.
    Further, each recursive call performs $d_{1}$ operations within the for
    loop in order to decrement each of $d_{2},d_{3},..,d_{d_{1}+1}$. The recurrence
    can thus be expressed as,
    \begin{align*}
      T(n) = \begin{cases}
                T(n - 1) + d_{1} &, \text{if n $>$ 0} \\
                1                &, \text{if n = 0}
             \end{cases}
    \end{align*}
    In the worst case each of the $n$ degrees wil be a maximum (i.e. $d_{1}$) at some point,
    therefore the total complexity of the recursive calls would be
    \begin{align*}
      \sum_{i=1}^{n} d_{i}
    \end{align*}

    \textbf{Total:} In summation,
    \begin{align*}
      T(n) & = O(nlgn) + \sum_{i=1}^{n} d_{i} & \\
           & = O(nlgn) + D & \\
           & = O(nlgn + D) &
    \end{align*}
  \end{lemma}
\end{question}

\end{document}
