\documentclass[12pt]{article}

\usepackage[margin=1in]{geometry}
\usepackage{amsmath,amsthm,amssymb}
\usepackage[none]{hyphenat}
\usepackage{mathtools}
\usepackage{verbatim}
\setlength{\parindent}{0pt}

\newcommand{\N}{\mathbb{N}}
\newcommand{\Z}{\mathbb{Z}}
\newcommand{\R}{\mathbb{R}}

\newenvironment{theorem}[2][Theorem]{\begin{trivlist}
\item[\hskip \labelsep {\bfseries #1}\hskip \labelsep {\bfseries #2.}]}{\end{trivlist}}
  \newenvironment{lemma}[2][Lemma]{\begin{trivlist}
\item[\hskip \labelsep {\bfseries #1}\hskip \labelsep {\bfseries #2.}]}{\end{trivlist}}
  \newenvironment{exercise}[2][Exercise]{\begin{trivlist}
\item[\hskip \labelsep {\bfseries #1}\hskip \labelsep {\bfseries #2.}]}{\end{trivlist}}
  \newenvironment{problem}[2][Problem]{\begin{trivlist}
\item[\hskip \labelsep {\bfseries #1}\hskip \labelsep {\bfseries #2.}]}{\end{trivlist}}
  \newenvironment{question}[2][Question]{\begin{trivlist}
\item[\hskip \labelsep {\bfseries #1}\hskip \labelsep {\bfseries #2.}]}{\end{trivlist}}
  \newenvironment{corollary}[2][Corollary]{\begin{trivlist}
\item[\hskip \labelsep {\bfseries #1}\hskip \labelsep {\bfseries #2.}]}{\end{trivlist}}
\DeclarePairedDelimiter{\ceil}{\lceil}{\rceil}

\begin{document}

\title{Assignment 1}%replace X with the appropriate number
\author{Quinn Perfetto, 104026025\\ %replace with your name
 60-454 Design and Analysis of Algorithms} %if necessary, replace with your course title

\maketitle

\begin{question}{1 (i)}
  We want to prove that when Algorithm Summation terminates its execution,
  \begin{align*}
    \texttt{$y = \sum_{j=0}^{n} a_{j}x^{j}$}
  \end{align*}
  \begin{proof}
    We shall apply induction to prove that after the m$th$ iteration of the
    for loop the following invariant holds true:
    \begin{align*}
      \texttt{$y = \sum_{j=0}^{m - 1} a_{n - j}x^{n - i - j}$}
    \end{align*}

    \text{(Induction Basis)}
    First, we note that y is initialized to 0 in Line 1. When $m=1$ and $i=n$
    in Line 3,
    \begin{align*}
      y & = a_{n} + x * y\\
        & = a_{n} + x * 0\\
        & = a_{n}\\
        & = a_{n}x^{0}\\
        & = \sum_{j=0}^{0} a_{n - j}x^{n - i - j}\\
    \end{align*}

    \text{(Induction Hypothesis)}
    Suppose for iteration $m - 1 < n$,
    $y = \sum_{j=0}^{m-2} a_{n - j}x^{n - i - j - 1}$

    \text{(Induction Step)}
    When Line 3 is executed for the m$th$ time,

    \begin{align*}
      y & = a_{n - i} + x\sum_{j=0}^{m-2} a_{n - j}x^{n - i - j - 1}\\
        & = a_{n - i} + \sum_{j=0}^{m-2} a_{n - j}x^{n - i - j} \\
        & = a_{n - i}x^{0} + \sum_{j=0}^{m-2} a_{n - j}x^{n - i - j}\\
        & & \text{(Induction Hypothesis)}\\
        & = \sum_{j=0}^{m-1} a_{n-j}x^{n - i - j}\\
    \end{align*}
    Therefore we can conclude that the invariant
    \texttt{$y = \sum_{j=0}^{m - 1} a_{n - j}x^{n - i - j}$} holds $\forall m > 0$. \\


  On the n+1$th$ iteration (where $i = 0$) we then have,
  \begin{align*}
    y & = \sum_{j=0}^{n} a_{n - j}x^{n - j}\\
      & = a_{n}x^{n} + a_{n-1}x^{n-1} + ... + a_{1}x^{1} + a_{0}x^{0}\\
      & = \sum_{j=0}^{n} a_{j}x^{j} & \text{(By reversing the order of the terms)}\\
  \end{align*}

  \end{proof}
\end{question}

\begin{question}{1 (ii)}
  \leavevmode \\
  \underline{Key Operation:} Multiplication of real numbers. \\
  \underline{Input size:} n (The size of the input array) \\ \\
  The Summation algorithm must perform the key operation on each of
  the $n$ elements of the input array.
  \begin{enumerate}
    \item Worst-case Time Complexity: $T(n) = n$
    \item Average-case Time Complexity: $T_{ave}(n) = n$
  \end{enumerate}
\end{question}

\begin{question}{2} $lgn \in \theta(lglg(n!))$
  \leavevmode \\ \\
  To prove that $lgn \in \theta(lglg(n!))$ we shall prove that
  $lgn \in O(lglg(n!)) \land lgn \in \Omega(lglg(n!))$.

  \begin{proof}
    First we shall prove that $lgn \in O(lglg(n!))$, that is we shall determine
    a $c \in \R^{+}$ and $n_{0} \in \N$ such that,
    \begin{align*}
      & lgn \leq clglg(n!)    & \forall n > n_{0} \\
      & lgn \leq lglg^{c}(n!) & \\
      & 2^{lgn} \leq 2^{lglg^{c}(n!)} &\\
      & n \leq lg^{c}n! &
    \end{align*}
    By letting c = 1, we obtain
    \begin{align*}
      & n \leq lg(n!) &\\
      & 2^{n} \leq n! &\\
    \end{align*}

    \begin{lemma}{1} $2^{n} < n!, \forall n \geq 4$
      \leavevmode \\
      We shall prove this by induction on $n$.

      \text{(Induction Basis)} When n = 4, $2^{4} < 4!$ = $16 < 24$. \\
      \text{(Induction Hypothesis)} Assume $2^{k} < k!, k \geq 4$. \\
      \text{(Induction Step)} We then have,
      \begin{align*}
        2^{k}(k + 1) & < k!(k + 1) &\\
                     & = (k + 1)! & (I)
      \end{align*}
      Also since $k \geq 4$,
      \begin{align*}
        2       & < k + 1        &\\
        2^{k+1} & < (k + 1)2^{k} & (II)\\
      \end{align*}

      By transitivity of (I) and (II) we obtain $2^{k+1} < (k+1)!$.\\
      Therefore by Induction $2^{n} < n!, \forall n \geq 4$.
    \end{lemma}

    Therefore for $c = 1$, and $n_{0} = 4$ we have
    $lg(n) \leq clglg(n!),\ \forall n > n_{0}$.\\
    Therefore $lg(n) \in O(lglg(n!))$.

    \leavevmode \\
    Now we shall prove that $lg(n) \in \Omega(lglg(n!))$, that is we shall
    determine a $c \in R^{+}$ and $n_{0} \in \N$ such that,
    \begin{align*}
      & lg(n) \geq clglg(n!) & \forall n > n_{0}\\
      & lg(n) \geq lglg^{c}(n!)&\\
      & 2^{lgn} \geq 2^{lglg^{c}(n!)} & \\
      & n \geq lg^{c}n! &\\ 
    \end{align*}
    By letting $c = \frac{1}{2}$ we obtain,
    \begin{align*}
      & n \geq \sqrt{lg(n!)} & \\
      & n^{2} \geq lg(n!)    & \\
    \end{align*}
    
    It can be seen that $lg(n!) \leq nlg(n)\ (III)$ via,
    \begin{align*}
      lg(n!) & = lg(n * (n-1) * (n-2) * ... * 1) & \text{(Definition of $n!$)}\\
             & = lg(n) + lg(n - 1) + ... + lg(1) & \text{$(lg(mn) = lg(m) + lg(n)$)}\\
             & \leq lg(n) + lg(n) + ... + lg(n)  &\\
             & = nlgn                            & 
    \end{align*}

    Additionally $\forall n > 1$,
    \begin{align*}
      2^{n} & > n      &\\
          n & > lg(n)  & \\
      n^{2} & > nlg(n) & (IV)\\
    \end{align*}

    By combining (III) and (IV) we have,
    $n^{2} > nlg(n) \geq lg(n!),\ \forall n > 1$.\\
    Therefore for $c = \frac{1}{2}$ and $n_{0} = 1$,
    $lg(n) \geq clglg(n!),\ \forall n > n_{0}$.\\
    Therefore $lg(n) \in O(lglg(n!)) \land lg(n) \in \Omega(lglg(n!))$.\\
    Therefore $lg(n) \in \theta(lglg(n!))$.

  \end{proof}
\end{question}

\begin{question}{3}
  I assert that the claim is false, and will provide a counter example such
  that $f \in O(g) \land h(f) \notin O(h(g))$.
  \begin{proof}
    Let
    \begin{align*}
      f(n) &= n^{2},\\
      g(n) &= n^{3},\\
      h(n) &= \frac{1}{n}
    \end{align*}

    \begin{lemma}{I} $f \in O(g)$, using Theorem 0.2
      \begin{align*}
        & \lim_{n\to\infty} \frac{n^{2}}{n^{3}} &  \\
        & = \lim_{n\to\infty} \frac{1}{n}       & \\
        & = 0 (\geq 0)                            & \text{(Theorem 0.2)}
      \end{align*}
      Therefore by Theorem 0.2 we have $f \in O(g)$.
    \end{lemma}

    Given the definition of $f$ and $h$ we can see that $h(f(n)) = \dfrac{1}{n^{2}}$.\\
    Given the definition of $g$ and $h$ we can see that $h(g(n)) = \dfrac{1}{n^{3}}$.\\
    All that remains is to prove that $\frac{1}{n^{2}}\notin O(\frac{1}{n^{3}})$.  Using Theorem 0.2,
    \begin{align*}
    & \lim_{n\to\infty} \frac{\frac{1}{n^{2}}}{\frac{1}{n^{3}}} &\\
        & = \lim_{n\to\infty} \frac{n^3}{n^2} &\\
        & = \lim_{n\to\infty} n &\\
        & = \infty (\notin \R^{+}) &\\
    \end{align*}
    Hence combining Lemma I with the above proof we have created a counter example
    such that,
    \begin{align*}
      f \in O(g) \land h(f) \notin O(h(g))
    \end{align*}
    Therefore the claim is false.
  \end{proof}
\end{question}

\begin{question}{4}
  I assert that the following function ordering respects the relationship
  $g1 = o(g2)$, $g2 = o(g3)$, ..., $g7 = o(g8)$.  Functions which are
  asymptotically equivalent are contained within \{ \}.
  \begin{align*}
    10^{100},\ \bigl\{\sum_{k=1}^{n} \frac{k^{2} + 2}{3k^{3} + 2k^{2} + 1},\ 
    \ lg(n) \bigr\},\ 10^{lglg(n)},\ 2^{\sqrt{2lg(n)}},\ 4^{lgn},\ n^{lgn},\ 2^{n}
  \end{align*}
  A series of proofs follow to confirm this ordering,

  \begin{proof} $n^{lgn} = o(2^{n})$
    \leavevmode \\ \\
    For any $c \in R^{+}$ define $n_{0} = \begin{cases}
                                               17 & if\ c \geq 1 \\
                                               \ceil{\frac{16}{c}} & if\ c < 1\\
                                          \end{cases}$\\ \\ \\
    \text{(i)} $n^{lgn} < 2^{n} \Rightarrow lg^{2}n < n$,
    which holds $\forall n > 16$.\\
    If $c \geq 1$ then $c2^{n} > 2^{n} > n^{lgn},\ \forall n \geq 17 = n_{0}$.\\

  \end{proof}

  \begin{proof} $4^{lgn} = o(n^{lgn})$
    \leavevmode \\ \\
    It can been seen that $4^{lgn} = n^{lg(4)} = n^{2}$.  Using Theorem 0.2,
    \begin{align*}
      &\lim_{n\to\infty} \frac{
        n^{2}
      }{
        n^{lgn}
      }&\\ \\
      = &\lim_{n\to\infty} \frac{
        \dfrac{n^{2}}{n^{2}}
      }{
        \dfrac{n^{lgn}}{n^{2}}
      }&\\ \\
      = & \lim_{n\to\infty} \frac{
        1
      }{n^{lgn - 2}}\\ \\
      = & 0
    \end{align*}
    Therefore $4^{lgn} = o(n^{lgn})$.
  \end{proof}

  \begin{proof} $2^{\sqrt{2lgn}} \in o(4^{lgn})$
    \leavevmode \\ \\
    It can be seen that $4^{lgn} = (2^{2})^{lgn} = 2^{2lgn}$.
    By letting $x$ represent $2lgn$ and applying Theorem 0.2 we have,
    \begin{align*}
      &\lim_{n\to\infty} \frac{
        2^{\sqrt{x}}
      }{
        2^{x}
      }&\\ \\
      & = \lim_{n\to\infty} \frac{
        \dfrac{
          2^{\sqrt{x}}
        }{
          2^{\sqrt{x}}
        }
      }{
        \dfrac{
          2^{x}
        }{
          2^{\sqrt{x}}
        }
      }&\\ \\
      & = \lim_{n\to\infty} \frac{1}{
        2^{x - \sqrt{x}}
      }&\\ \\
      & = 0&\\
    \end{align*}
    Therefore $2^{\sqrt{2lgn}} = o(4^{lgn})$.
  \end{proof}
  
  \begin{proof} $10^{lglg(n)} = o(2^{\sqrt{2lgn}})$
  \end{proof}

  \begin{proof} $lg(n) = o(10^{lglg(n)})$
    \leavevmode \\ \\
    It can be seen that $10^{lglg(n)} = lg^{lg10}n$, where $lg10 > 1 = 1 + \epsilon$ for some $\epsilon > 2$.
    Using Theorem 0.2 we have,
    \begin{align*}
      &\lim_{n\to\infty} \frac{
        lg(n)
      }{
        lg^{1 + \epsilon}n
      }&\\ \\
      = & \lim_{n\to\infty} \frac{
        \dfrac{lg(n)}{lg(n)}
      }{
        \dfrac{lg^{1 + \epsilon}n}{lg(n)}
      }&\\ \\
      = & \lim_{n\to\infty} \frac{
        1
      }{
        lg^{\epsilon}n
      }&\\ \\
      = & 0 &
    \end{align*}

    Therefore $lg(n) = o(10^{lglg(n)})$.
  \end{proof}

  \begin{proof} $\sum_{k=1}^{n}\frac{k^{2}}{3k^{3} + 2k^{2} + 1} = \theta(lgn)$
    \leavevmode \\ \\
    We shall first establish that,
    \begin{align*}
      \sum_{k=1}^{n}\frac{k^{2} + 2}{3k^{3} + 2k^{2} + 1} = \theta(\sum_{k=1}^{n}\frac{1}{k})
    \end{align*}

    Since $\sum_{k=1}^{\infty} \frac{1}{k} = \infty^{+}$ and

    \begin{align*}
      & \lim_{n\to\infty} \frac{
        \frac{n^{2} + 2}{3n^{3} + 3n^{2} + 1}
      }{
        \frac{1}{n}
      } &\\
        = & \lim_{n\to\infty} \frac{n^{3} + 2n}{3n^{3} + 3n^{2} + 1} &\\
        = & \lim_{n\to\infty} \frac{
              \frac{n^{3}}{n^{3}} + \frac{2n}{n^{3}}
            }{
              \frac{3n^{3}}{n^{3}} + \frac{3n^{2}}{n^{3}} + \frac{1}{n^{3}}
            } &\\
        = & \frac{1}{3}
    \end{align*}
    By the The Stolz-Cesaro Theorem,
    \begin{align*}
      & \lim_{n\to\infty} \frac{
        \frac{n^{2} + 2}{3n^{3} + 3n^{2} + 1}
      }{
        \frac{1}{n}
      } &\\
      = & \lim_{n\to\infty} \frac{
        \sum_{k=1}^{n} \frac{k^{2} + 2}{3k^{3} + 2k^{2} + 1}
      }{
        \sum_{k=1}^{n} \frac{1}{k}
      } &\\
        = & \frac{1}{3} & (> 0)
    \end{align*}

    Therefore by Theorem 0.2 $\sum_{k=1}^{n}\frac{k^{2} + 2}{3k^{3} + 2k^{2} + 1} = \theta(\sum_{k=1}^{n}\frac{1}{k})$ (I).\\ \\
    Additionally we have $\sum_{k=1}^{n} \frac{1}{k} = \theta(lgn)$ via,
    \begin{align*}
      & \lim_{n\to\infty} \frac{
        \sum_{k=1}^{n} \frac{1}{k}
      }{lgn} &\\
      = & \lim_{n\to\infty} \frac{
        ln(n) + \gamma + \frac{1}{2n} + o(\frac{1}{n})
      }{lgn} & \text{(Hint)}\\
      = & \lim_{n\to\infty} \frac{
        ln(n) + \gamma
      }{lgn} & \\
      = & \lim_{n\to\infty} \frac{
        ln(n) + \gamma
      }{
        \frac{ln(n)}{ln(2)}
      } & \\
      = & \lim_{n\to\infty} \frac{
        \frac{ln(n)}{ln(n)} + \frac{\gamma}{ln(n)}
      }{
        \frac{ln(n)}{ln(n)ln(2)}
      } & \\
      = & ln(2) & (> 0)\\
    \end{align*}
    Therefore by Theorem 0.2 $\sum_{k=1}^{n}\frac{1}{k} = \theta(lgn)$ (II).\\ \\
    Given (I) and (II) and the transitivity of $\theta$ we obtain $\sum_{k=1}^{n}\frac{k^{2} + 2}{3k^{3} + 2k^{2} + 1} = \theta(lgn)$.
  \end{proof}

  \begin{proof} $10^{100} = o(lgn)$
    \leavevmode \\ \\
    Since it was shown that $lgn$ and $\sum_{k=1}^{n}\frac{k^{2} + 2}{3k^{3} + 2k^{2} + 1}$
    are asymptotically equivalent, it is sufficient to show that $10^{100}$ is little-o
    of either one to confirm the ordering.  To this end we apply Theorem 0.2,
    \begin{align*}
        & \lim_{n\to\infty} \frac{10^{100}}{lgn} &\\
      = & \lim_{n\to\infty} \frac{0}{\frac{lg(e)}{n}} & \text{(L'Hopital's rule)}\\
      = & 0 &
    \end{align*}
    Therefore by Theorem 0.2 we have $10^{100} = o(lgn)$.
  \end{proof}
\end{question}

\begin{question}{5 (a)} $T(n) = 9T(\frac{n}{3}) + n^{2}lg(n) + 2n$
  \leavevmode \\
  \begin{proof}
  Using the general formula for recurrences we note that,
  \begin{align*}
    a = 9, b = 3, f(n) = n^{2}lg(n) + 2n
  \end{align*}

    \begin{lemma}{I} $f(n) \in \theta(n^{2}lg(n))$, using Theorem 0.2
      \begin{align*}
        &\lim_{n\to\infty} \frac{
        n^{2}lg(n) + 2n
      }{
        n^{2}lg(n)
      }\\ \\
      & = \lim_{n\to\infty} \frac{
        \dfrac{
          n^{2}lg(n) + 2n
        }{n^{2}}
      }{
        \dfrac{
          n^{2}lg(n)
        }{n^{2}}
      }\\ \\
      & = \lim_{n\to\infty} \frac{
        lg(n) + \frac{2}{n}
      }{
        lg(n)
      }\\ \\
      & = \lim_{n\to\infty} \frac{
        lg(n)
      }{
        lg(n)
      }\\ \\
      & = 1 (> 0)\\
      \end{align*}
    \end{lemma}

    By Lemma I we have $f(n) \in \theta(n^{2}lg(n)) = \theta(n^{log_{b}a}log^{k}n)$ where $k = 1 (\geq 0)$.
    Therefore using Case 2 of the general recurrence formula we have,
    \begin{align*}
      T(n) \in \theta(n^{2}lg^{2}n)
    \end{align*}
  \end{proof}
\end{question}

\begin{question}{5 (b)} $T(n) = 3T(\frac{n}{3}) + \sqrt{n}$
  \leavevmode \\
  \begin{proof}
    Using the general formula for recurrences we note that,
    \begin{align*}
      a = 3, b = 3, f(n) = \sqrt{n}
    \end{align*}
    Therefore, $f(n) = \sqrt{n} = O(n^{1 - \frac{1}{2}}) = O(n^{log_{b}a - \epsilon})$
    where $\epsilon = \frac{1}{2} > 0$.
    Therefore using Case 1 of the general recurrence theorem we have,
    \begin{align*}
      T(n) \in \theta(n^{log_{3}3}) = \theta(n)
    \end{align*}
  \end{proof}
\end{question}

\begin{question}{5 (c)} $T(n) = 8T(\frac{n}{4}) + n^{2}lg^{2}n$
  \begin{proof}
    Using the general formula for recurrences we note that,
    \begin{align*}
      a = 8, b = 4, f(n) = n^{2}lg^{2}n
    \end{align*}

    \begin{lemma}{I} $f(n) \in \Omega(n^{2})$ using Theorem 0.2,
      \begin{align*}
        & \lim_{x\to\infty} \frac{
            n^{2}lg^{2}n
          }{
            n^{2}
          } &\\ \\
        = & \lim_{x\to\infty} \frac{
          \dfrac{n^{2}lg^{2}n}{n^{2}}
        }{
          \dfrac{n^{2}}{n^{2}}
        } &\\ \\
        = & \lim_{x\to\infty} \frac{
          lg^{2}n
        }{
          1
        }&\\ \\
        & = \infty &
      \end{align*}
      Hence by Theorem 0.2 $f(n) \in \Omega(n^{2})$.
    \end{lemma}

    By Lemma I we have $f(n) \in \Omega(n^{log_{b}a + \epsilon})$ where
    $\epsilon = \frac{1}{2} > 0$.
    Moreover, for sufficiently large $n$,
    \begin{align*}
      af(\frac{n}{b}) & = 8(\frac{n}{4})^{2}lg^{2}\frac{n}{4}&\\
                      & = 8(\frac{n^{2}}{16})lg^{2}\frac{n}{4}&\\
                      & = \frac{1}{2}(n^{2}lg^{2}\frac{n}{4})&\\
                      & = \frac{1}{2} \bigl(n^{2}(lgn - 2)^{2}\bigr) &\\
                      & = \frac{1}{2} \bigl(n^{2}(lg^{2}n - 4lgn + 4)\bigr) &\\
                      & = \frac{1}{2}n^{2}log^{2}n - 2n^{2}lgn + 2n^{2}&\\
                      & \leq \frac{1}{2}n^{2}log^{2}n & \text{when $n \geq 2$}\\
                      & = cn^{2}log^{2}n & \text{where $0 < c = \frac{1}{2} < 1$}\\
    \end{align*}

    By Case 3 of the general recurrence theorem we thus have $T(n) \in \theta(n^{2}lg^{2}n)$.
  \end{proof}
\end{question}

\begin{question}{6} $T(n)  = T(\frac{n}{2} + 7)$
  \begin{proof}
    We guess that $T(n) = O(n^{2})$.\\
    (Induction Hypothesis) We first assume that $T(k) \leq ck^{2},\ \forall k < n$ (I).\\
    (Induction Step) Then, when $n > 14$
    \begin{align*}
      & n > 14 \Rightarrow 2n > 14 + n \Rightarrow \frac{n}{2} + 7 < n &\\
      & \Rightarrow T(\frac{n}{2} + 7) \leq c(\frac{n}{2} + 7)^{2} & \text{(by I)}\\
    \end{align*}

    Therefore, for $ n > 14$,
    \begin{align*}
      T(n) & = T(\dfrac{n}{2} + 7) + n^{2}               &\\
           & \leq c(\dfrac{n}{2} + 7)^{2} + n^{2}        &\\
           & \leq \dfrac{1}{4}cn^{2} + 7cn + 49c + n^{2} &\\
    \end{align*}

    For $c \geq 4$,
    \begin{align*}
      & c \geq 4           &\\
      & \frac{c}{4} \geq 1 &\\
      & \frac{c}{4} + \frac{c}{4} \geq 1 + \frac{c}{4} &\\
      & \frac{c}{2} \geq \frac{c}{4} + 1 &\\
      & \frac{1}{2}cn^{2} \geq \frac{1}{4}cn^{2} + n^{2} &\\
    \end{align*}

    Therefore,
    \begin{align*}
      T(n) & \leq \dfrac{1}{4}cn^{2} + 7cn + 49c + n^{2} &\\
           & \leq \frac{1}{2}cn^{2} + 7cn + 49c & (c \geq 4)\\
           & \leq cn^{2} & (n \geq 20)
    \end{align*}
    Hence, $T(n) \leq cn,\ \forall n \geq 20$ and any $c \geq 4$.\\ \\
    (Inductive Basis) Let $c' = max(\{T(n)/n^{2}\ |\ 17 \leq n \leq 19\}\ \bigcup\ \{4\})$.\\
    Then $T(n) \leq c'n^{2},\ \forall n,\ 17 \leq n \leq 19$.\\
    Hence $T(n) \leq c'n^{2},\ \forall n \geq 17$.\\
    i.e. $T(n) = O(n^{2})$.\\
  \end{proof}
\end{question}

\end{document}
