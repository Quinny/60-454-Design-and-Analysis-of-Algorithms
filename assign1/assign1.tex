\documentclass[12pt]{article}

\usepackage[margin=1in]{geometry}
\usepackage{amsmath,amsthm,amssymb}
\usepackage[none]{hyphenat}
\setlength{\parindent}{0pt}

\newcommand{\N}{\mathbb{N}}
\newcommand{\Z}{\mathbb{Z}}
\newcommand{\R}{\mathbb{R}}

\newenvironment{theorem}[2][Theorem]{\begin{trivlist}
\item[\hskip \labelsep {\bfseries #1}\hskip \labelsep {\bfseries #2.}]}{\end{trivlist}}
  \newenvironment{lemma}[2][Lemma]{\begin{trivlist}
\item[\hskip \labelsep {\bfseries #1}\hskip \labelsep {\bfseries #2.}]}{\end{trivlist}}
  \newenvironment{exercise}[2][Exercise]{\begin{trivlist}
\item[\hskip \labelsep {\bfseries #1}\hskip \labelsep {\bfseries #2.}]}{\end{trivlist}}
  \newenvironment{problem}[2][Problem]{\begin{trivlist}
\item[\hskip \labelsep {\bfseries #1}\hskip \labelsep {\bfseries #2.}]}{\end{trivlist}}
  \newenvironment{question}[2][Question]{\begin{trivlist}
\item[\hskip \labelsep {\bfseries #1}\hskip \labelsep {\bfseries #2.}]}{\end{trivlist}}
  \newenvironment{corollary}[2][Corollary]{\begin{trivlist}
\item[\hskip \labelsep {\bfseries #1}\hskip \labelsep {\bfseries #2.}]}{\end{trivlist}}

\begin{document}

\title{Assignment 1}%replace X with the appropriate number
\author{Quinn Perfetto, 104026025\\ %replace with your name
 60-454 Design and Analysis of Algorithms} %if necessary, replace with your course title

\maketitle

\begin{question}{1. (i)}
  We want to prove that when Algorithm Summation terminates its execution,
  \begin{align*}
    \texttt{$y = \sum_{j=0}^{n} a_{j}x^{j}$}
  \end{align*}
  \begin{proof}
    We shall apply induction to prove that after the m$th$ iteration of the
    for loop the following invariant holds true:
    \begin{align*}
      \texttt{$y = \sum_{j=0}^{m - 1} a_{n - j}x^{n - i - j}$}
    \end{align*}

    \text{(Induction basis)}
    First, we note that y is initialized to 0 in Line 1. When $m=1$ and $i=n$
    in Line 3,
    \begin{align*}
      y & = a_{n} + x * y\\
        & = a_{n} + x * 0\\
        & = a_{n}\\
        & = a_{n}x^{0}\\
        & = \sum_{j=0}^{0} a_{n - j}x^{n - i - j}\\
    \end{align*}

    \text{(Induction Hypothesis)}
    Suppose for iteration $m - 1 < n$,
    $y = \sum_{j=0}^{m-2} a_{n - j}x^{n - i - j - 1}$

    \text{(Induction Step)}
    When Line 3 is executed for the m$th$ time,

    \begin{align*}
      y & = a_{n - i} + x\sum_{j=0}^{m-2} a_{n - j}x^{n - i - j - 1}\\
        & = a_{n - i} + \sum_{j=0}^{m-2} a_{n - j}x^{n - i - j} \\
        & = a_{n - i}x^{0} + \sum_{j=0}^{m-2} a_{n - j}x^{n - i - j}\\
        & & \text{(Induction Hypothesis)}\\
        & = \sum_{j=0}^{m-1} a_{n-j}x^{n - i - j}\\
    \end{align*}
    Therefore we can conclude that the invariant
    \texttt{$y = \sum_{j=0}^{m - 1} a_{n - j}x^{n - i - j}$} holds $\forall m > 0$. \\


  On the n+1$th$ iteration (where $i = 0$) we then have,
  \begin{align*}
    y & = \sum_{j=0}^{n} a_{n - j}x^{n - j}\\
      & = a_{n}x^{n} + a_{n-1}x^{n-1} + ... + a_{1}x^{1} + a_{0}x^{0}\\
      & = \sum_{j=0}^{n} a_{j}x^{j} & \text{(By reversing the order of the terms)}\\
  \end{align*}

  \end{proof}
\end{question}

\begin{question}{1. (ii)}
  \leavevmode \\
  \underline{Key Operation:} Multiplication of real numbers. \\
  \underline{Input size:} n (The size of the input array) \\ \\
  The Summation algorithm must perform the key operation on each of
  the $n$ elements of the input array.
  \begin{enumerate}
    \item Worst-case Time Complexity: $T(n) = n$
    \item Average-case Time Complexity: $T_{ave}(n) = n$
  \end{enumerate}
\end{question}

\begin{question}{3}
  I assert that the claim is false, and will provide a counter example such
  that $f \in O(g) \land h(f) \notin O(h(g))$.
  \begin{proof}
    Let
    \begin{align*}
      f(n) &= n^{2},\\
      g(n) &= n^{3},\\
      h(n) &= \frac{1}{n}
    \end{align*}

    \begin{lemma}{I} $f \in O(g)$, using Theorem 0.2
      \begin{align*}
        & \lim_{n\to\infty} \frac{n^{2}}{n^{3}} &  \\
        & = \lim_{n\to\infty} \frac{1}{n}       & \\
        & = 0 (>= 0)                            & \text{(Theorem 0.2)}
      \end{align*}
    \end{lemma}

    Given the defintion of $f$ and $h$ we can see that $h(f(n)) = \dfrac{1}{n^{2}}$.\\
    Given the defintion of $g$ and $h$ we can see that $h(g(n)) = \dfrac{1}{n^{3}}$.\\
    All that remains is to prove that $\frac{1}{n^{2}}\notin O(\frac{1}{n^{3}})$,
    \begin{align*}
    & \lim_{n\to\infty} \frac{\frac{1}{n^{2}}}{\frac{1}{n^{3}}} &\\
        & = \lim_{n\to\infty} \frac{n^3}{n^2} &\\
        & = \lim_{n\to\infty} n &\\
        & = \infty (\notin \R^{+}) &\\
    \end{align*}
    Hence combining Lemma I with the above proof we have created a counter example
    such that,
    \begin{align*}
      f \in O(g) \land h(f) \notin O(h(g))
    \end{align*}
  \end{proof}
\end{question}

\end{document}
