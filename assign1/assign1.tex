\documentclass[12pt]{article}

\usepackage[margin=1in]{geometry}
\usepackage{amsmath,amsthm,amssymb}
\usepackage[none]{hyphenat}
\setlength{\parindent}{0pt}

\newcommand{\N}{\mathbb{N}}
\newcommand{\Z}{\mathbb{Z}}
\newcommand{\R}{\mathbb{R}}

\newenvironment{theorem}[2][Theorem]{\begin{trivlist}
\item[\hskip \labelsep {\bfseries #1}\hskip \labelsep {\bfseries #2.}]}{\end{trivlist}}
  \newenvironment{lemma}[2][Lemma]{\begin{trivlist}
\item[\hskip \labelsep {\bfseries #1}\hskip \labelsep {\bfseries #2.}]}{\end{trivlist}}
  \newenvironment{exercise}[2][Exercise]{\begin{trivlist}
\item[\hskip \labelsep {\bfseries #1}\hskip \labelsep {\bfseries #2.}]}{\end{trivlist}}
  \newenvironment{problem}[2][Problem]{\begin{trivlist}
\item[\hskip \labelsep {\bfseries #1}\hskip \labelsep {\bfseries #2.}]}{\end{trivlist}}
  \newenvironment{question}[2][Question]{\begin{trivlist}
\item[\hskip \labelsep {\bfseries #1}\hskip \labelsep {\bfseries #2.}]}{\end{trivlist}}
  \newenvironment{corollary}[2][Corollary]{\begin{trivlist}
\item[\hskip \labelsep {\bfseries #1}\hskip \labelsep {\bfseries #2.}]}{\end{trivlist}}

\begin{document}

\title{Assignment 1}%replace X with the appropriate number
\author{Quinn Perfetto, 104026025\\ %replace with your name
 60-454 Design and Analysis of Algorithms} %if necessary, replace with your course title

\maketitle

\begin{question}{1 (i)}
  We want to prove that when Algorithm Summation terminates its execution,
  \begin{align*}
    \texttt{$y = \sum_{j=0}^{n} a_{j}x^{j}$}
  \end{align*}
  \begin{proof}
    We shall apply induction to prove that after the m$th$ iteration of the
    for loop the following invariant holds true:
    \begin{align*}
      \texttt{$y = \sum_{j=0}^{m - 1} a_{n - j}x^{n - i - j}$}
    \end{align*}

    \text{(Induction basis)}
    First, we note that y is initialized to 0 in Line 1. When $m=1$ and $i=n$
    in Line 3,
    \begin{align*}
      y & = a_{n} + x * y\\
        & = a_{n} + x * 0\\
        & = a_{n}\\
        & = a_{n}x^{0}\\
        & = \sum_{j=0}^{0} a_{n - j}x^{n - i - j}\\
    \end{align*}

    \text{(Induction Hypothesis)}
    Suppose for iteration $m - 1 < n$,
    $y = \sum_{j=0}^{m-2} a_{n - j}x^{n - i - j - 1}$

    \text{(Induction Step)}
    When Line 3 is executed for the m$th$ time,

    \begin{align*}
      y & = a_{n - i} + x\sum_{j=0}^{m-2} a_{n - j}x^{n - i - j - 1}\\
        & = a_{n - i} + \sum_{j=0}^{m-2} a_{n - j}x^{n - i - j} \\
        & = a_{n - i}x^{0} + \sum_{j=0}^{m-2} a_{n - j}x^{n - i - j}\\
        & & \text{(Induction Hypothesis)}\\
        & = \sum_{j=0}^{m-1} a_{n-j}x^{n - i - j}\\
    \end{align*}
    Therefore we can conclude that the invariant
    \texttt{$y = \sum_{j=0}^{m - 1} a_{n - j}x^{n - i - j}$} holds $\forall m > 0$. \\


  On the n+1$th$ iteration (where $i = 0$) we then have,
  \begin{align*}
    y & = \sum_{j=0}^{n} a_{n - j}x^{n - j}\\
      & = a_{n}x^{n} + a_{n-1}x^{n-1} + ... + a_{1}x^{1} + a_{0}x^{0}\\
      & = \sum_{j=0}^{n} a_{j}x^{j} & \text{(By reversing the order of the terms)}\\
  \end{align*}

  \end{proof}
\end{question}

\begin{question}{1 (ii)}
  \leavevmode \\
  \underline{Key Operation:} Multiplication of real numbers. \\
  \underline{Input size:} n (The size of the input array) \\ \\
  The Summation algorithm must perform the key operation on each of
  the $n$ elements of the input array.
  \begin{enumerate}
    \item Worst-case Time Complexity: $T(n) = n$
    \item Average-case Time Complexity: $T_{ave}(n) = n$
  \end{enumerate}
\end{question}

\begin{question}{3}
  I assert that the claim is false, and will provide a counter example such
  that $f \in O(g) \land h(f) \notin O(h(g))$.
  \begin{proof}
    Let
    \begin{align*}
      f(n) &= n^{2},\\
      g(n) &= n^{3},\\
      h(n) &= \frac{1}{n}
    \end{align*}

    \begin{lemma}{I} $f \in O(g)$, using Theorem 0.2
      \begin{align*}
        & \lim_{n\to\infty} \frac{n^{2}}{n^{3}} &  \\
        & = \lim_{n\to\infty} \frac{1}{n}       & \\
        & = 0 (\geq 0)                            & \text{(Theorem 0.2)}
      \end{align*}
    \end{lemma}

    Given the defintion of $f$ and $h$ we can see that $h(f(n)) = \dfrac{1}{n^{2}}$.\\
    Given the defintion of $g$ and $h$ we can see that $h(g(n)) = \dfrac{1}{n^{3}}$.\\
    All that remains is to prove that $\frac{1}{n^{2}}\notin O(\frac{1}{n^{3}})$.  Using Theorem 0.2,
    \begin{align*}
    & \lim_{n\to\infty} \frac{\frac{1}{n^{2}}}{\frac{1}{n^{3}}} &\\
        & = \lim_{n\to\infty} \frac{n^3}{n^2} &\\
        & = \lim_{n\to\infty} n &\\
        & = \infty (\notin \R^{+}) &\\
    \end{align*}
    Hence combining Lemma I with the above proof we have created a counter example
    such that,
    \begin{align*}
      f \in O(g) \land h(f) \notin O(h(g))
    \end{align*}
  \end{proof}
\end{question}

\begin{question}{4}
  I assert that the following function ordering respects the relationship
  $g1 = o(g2)$, $g2 = o(g3)$, ..., $g7 = o(g8)$.
  \begin{align*}
    10^{100},\ weirdsum,\ lg(n),\ 10^{lglg(n)},\ 2^{\sqrt{2lg(n)}},\ 4^{lgn},\ n^{lgn},\ 2^{n}
  \end{align*}
  A series of proofs follow to confirm this ordering,

  \begin{proof} $n^{lgn} = o(2^{n})$
    \leavevmode \\ \\
    Given that $n^{k} = o(2^{n}), \forall k > 0$ was proven on Page 48 of Chapter 0,
    and $lgn > 0, \forall n > 1$ we have $n^{lgn} = o(2^{n})$.
  \end{proof}

  \begin{proof} $4^{lgn} = o(n^{lgn})$
    \leavevmode \\ \\
    It can been seen that $4^{lgn} = n^{lg(4)} = n^{2}$.  Using Theorem 0.2,
    \begin{align*}
      &\lim_{n\to\infty} \frac{
        n^{2}
      }{
        n^{lgn}
      }&\\ \\
      = &\lim_{n\to\infty} \frac{
        \dfrac{n^{2}}{n^{2}}
      }{
        \dfrac{n^{lgn}}{n^{2}}
      }&\\ \\
      = & \lim_{n\to\infty} \frac{
        1
      }{n^{lgn - 2}}\\ \\
      = & 0
    \end{align*}
    Therefore $4^{lgn} = o(n^{lgn})$.
  \end{proof}

  \begin{proof} $2^{\sqrt{2lgn}} \in o(4^{lgn})$
    \leavevmode \\ \\
    It can be seen that $4^{lgn} = (2^{2})^{lgn} = 2^{2lgn}$.
    By letting $x$ represent $2lgn$ and applying Theorem 0.2 we have,
    \begin{align*}
      &\lim_{n\to\infty} \frac{
        2^{\sqrt{x}}
      }{
        2^{x}
      }&\\ \\
      & = \lim_{n\to\infty} \frac{
        \dfrac{
          2^{\sqrt{x}}
        }{
          2^{\sqrt{x}}
        }
      }{
        \dfrac{
          2^{x}
        }{
          2^{\sqrt{x}}
        }
      }&\\ \\
      & = \lim_{n\to\infty} \frac{1}{
        2^{x - \sqrt{x}}
      }&\\ \\
      & = 0&\\
    \end{align*}
    Therefore $2^{\sqrt{2lgn}} = o(4^{lgn})$.
  \end{proof}

  \begin{proof} $lg(n) = o(10^{lglg(n)})$
    \leavevmode \\ \\
    It can be seen that $10^{lglg(n)} = lg^{lg10}n$, where $lg10 > 1 = 1 + \epsilon$ for some $\epsilon > 2$.
    Using Theorem 0.2 we have,
    \begin{align*}
      &\lim_{n\to\infty} \frac{
        lg(n)
      }{
        lg^{1 + \epsilon}n
      }&\\ \\
      = & \lim_{n\to\infty} \frac{
        \dfrac{lg(n)}{lg(n)}
      }{
        \dfrac{lg^{1 + \epsilon}n}{lg(n)}
      }&\\ \\
      = & \lim_{n\to\infty} \frac{
        1
      }{
        lg^{\epsilon}n
      }&\\ \\
      = & 0 &
    \end{align*}

    Therefore $lg(n) = o(10^{lglg(n)})$.
  \end{proof}
\end{question}

\begin{question}{5 (a)} $T(n) = 9T(\frac{n}{3}) + n^{2}lg(n) + 2n$
  \leavevmode \\
  \begin{proof}
  Using the general formula for recurrences we note that,
  \begin{align*}
    a = 9, b = 3, f(n) = n^{2}lg(n) + 2n
  \end{align*}

    \begin{lemma}{I} $f(n) \in \theta(n^{2}lg(n))$, using Theorem 0.2
      \begin{align*}
        &\lim_{n\to\infty} \frac{
        n^{2}lg(n) + 2n
      }{
        n^{2}lg(n)
      }\\ \\
      & = \lim_{n\to\infty} \frac{
        \dfrac{
          n^{2}lg(n) + 2n
        }{n^{2}}
      }{
        \dfrac{
          n^{2}lg(n)
        }{n^{2}}
      }\\ \\
      & = \lim_{n\to\infty} \frac{
        lg(n) + \frac{2}{n}
      }{
        lg(n)
      }\\ \\
      & = \lim_{n\to\infty} \frac{
        lg(n)
      }{
        lg(n)
      }\\ \\
      & = 1 (> 0)\\
      \end{align*}
    \end{lemma}

    By Lemma I we have $f(n) \in \theta(n^{2}lg(n)) = \theta(n^{log_{b}a}log^{k}n)$ where $k = 1 (\geq 0)$.
    Therefore using Case 2 of the general recurrence forumla we have,
    \begin{align*}
      T(n) \in \theta(n^{2}lg^{2}n)
    \end{align*}
  \end{proof}
\end{question}

\begin{question}{5 (b)} $T(n) = 3T(\frac{n}{3}) + \sqrt{n}$
  \leavevmode \\
  \begin{proof}
    Using the general formula for recurrences we note that,
    \begin{align*}
      a = 3, b = 3, f(n) = \sqrt{n}
    \end{align*}

    \begin{lemma}{1} $f(n) \in O(n^{log_{3}2})$
      \begin{align*}
        &\frac{1}{2} \leq log_{3}2  &\\
        &\sqrt{n} \leq n^{log_{3}2} & (n \geq 0)
      \end{align*}
      Therefore for c = 1, and $n_{0} = 0$ we have,
      \begin{align*}
        \sqrt{n} <= cn^{log_{3}2}, \forall n_{0} > 0
      \end{align*}
      Hence $f(n) \in O(n^{log_{3}2})$.
    \end{lemma}

    By Lemma I we have $f(n) \in O(n^{log_{b}a - \epsilon})$ where $\epsilon = 1 (>0)$.
    Therefore using Case 1 of the general recurrence forumla we have,
    \begin{align*}
      T(n) \in \theta(n^{log_{3}3}) = \theta(n)
    \end{align*}
  \end{proof}
\end{question}

\begin{question}{5 (c)} $T(n) = 8T(\frac{n}{4}) + n^{2}lg^{2}n$
  \begin{proof}
    Using the general formula for recurrences we note that,
    \begin{align*}
      a = 8, b = 4, f(n) = n^{2}lg^{2}n
    \end{align*}

    \begin{lemma}{I} $f(n) \in \Omega(n^{log_{4}9})$
      \begin{align*}
        log_{4}9     & \leq 2&\\
        n^{log_{4}9} & \leq n^{2} &\\
                     & \leq n^{2}lg^{2}n & (n \geq 2)
      \end{align*}
      Therefore for $c = 1$ and $n_{0} = 2$ we have,
      \begin{align*}
        n^{2}lg^{2}n \geq cn^{log_{4}9}
      \end{align*}
      Hence $f(n) \in \Omega(n^{log_{4}9})$.

      By Lemma I we have $f(n) \in \Omega(n^{log_{b}a + \epsilon})$ where $\epsilon = 1$.
      Moreover, for sufficiently large $n$,
      \begin{align*}
        af(\frac{n}{b} 1
      \end{align*}
    \end{lemma}
  \end{proof}
\end{question}

\end{document}
