\documentclass[12pt]{article}

\usepackage[margin=1in]{geometry}
\usepackage{amsmath,amsthm,amssymb}
\usepackage[none]{hyphenat}
\usepackage{mathtools}
\usepackage{verbatim}
\usepackage[ruled]{algorithm2e}
\setlength{\parindent}{0pt}

\newcommand{\N}{\mathbb{N}}
\newcommand{\Z}{\mathbb{Z}}
\newcommand{\R}{\mathbb{R}}

\newenvironment{theorem}[2][Theorem]{\begin{trivlist}
\item[\hskip \labelsep {\bfseries #1}\hskip \labelsep {\bfseries #2.}]}{\end{trivlist}}
  \newenvironment{lemma}[2][Lemma]{\begin{trivlist}
\item[\hskip \labelsep {\bfseries #1}\hskip \labelsep {\bfseries #2.}]}{\end{trivlist}}
  \newenvironment{exercise}[2][Exercise]{\begin{trivlist}
\item[\hskip \labelsep {\bfseries #1}\hskip \labelsep {\bfseries #2.}]}{\end{trivlist}}
  \newenvironment{problem}[2][Problem]{\begin{trivlist}
\item[\hskip \labelsep {\bfseries #1}\hskip \labelsep {\bfseries #2.}]}{\end{trivlist}}
  \newenvironment{question}[2][Question]{\begin{trivlist}
\item[\hskip \labelsep {\bfseries #1}\hskip \labelsep {\bfseries #2.}]}{\end{trivlist}}
  \newenvironment{corollary}[2][Corollary]{\begin{trivlist}
\item[\hskip \labelsep {\bfseries #1}\hskip \labelsep {\bfseries #2.}]}{\end{trivlist}}
\DeclarePairedDelimiter{\ceil}{\lceil}{\rceil}
\DeclarePairedDelimiter{\floor}{\lfloor}{\rfloor}
\DeclarePairedDelimiter\abs{\lvert}{\rvert}%

\SetKwInput{KwInput}{Input}
\SetKwInput{KwOutput}{Output}
\SetKwInput{KwLet}{Let}
\SetKwProg{Fn}{Function}{}{}

\begin{document}

\title{Assignment 4}%replace X with the appropriate number
\author{Quinn Perfetto, 104026025\\ %replace with your name
 60-454 Design and Analysis of Algorithms} %if necessary, replace with your course title

\maketitle

\begin{question}{1}
  \begin{proof} First we show that $\Pi \in \textbf{NP}$ \\ \\
    Let $\pi \in \Pi$. \\ \\
    As every $\Pi$ logical expression is a CNF logical expression with exactly
    three literals, \\ $\pi \in 3SAT$. \\ \\
    Since 3SAT is \textbf{NP}-complete, there exists a polynomial time nondeterministic
    algorithm $N$ solving 3SAT. \\ \\
    Now,
    \begin{align*}
      \pi  & \text{ is a yes-instance of } \Pi \\
      \iff & \pi \text{ is a yes-instance of } 3SAT \\
      \iff & \text{algorithm } N \text{ outputs a yes on input } \pi
    \end{align*}
    Therefore $\Pi \in \textbf{NP}$. \\

    Next, we prove that $3SAT \alpha \Pi$. \\

    Let $C = c_{1} \land c_{2} \land ... \land c_{m}$ be a CNF logical expression
    in which each $c_{i}$ contains exactly three literals, i.e. $C$ is a problem instance of 3SAT.
    Let $U = \{u_{1}, u_{2}, ... , u_{n}\}$ be the set of variables in $C$. \\

    For each $c_{i}$, $1 \leq i \leq m$ we shall construct a $\Pi$ logical expression
    $c'_{i} \in \{x_{i}, y_{i}, z_{i}, a_{i}, b_{i}, c_{i}, d_{i}\}$
    such that the $\Pi$ CNF expression $C' = c'_{1} \land c'_{2} \land ... \land c'_{m}$
    is satisfiable with exactly 1 literal per clause being assigned a true value
    if and only if $C$ is satisfiable. \\

    Let $c_{j} = x_{j} \vee y_{j} \vee z_{j}$. The corresponding $c'_{j}$ is
    defined as
    \begin{align*}
      (\neg x_{j} \vee a_{j} \vee b_{j}) \land (y_{j} \vee b_{j} \vee c_{j}) \land (\neg z_{j} \land c_{j} \land d_{j})
    \end{align*}
    where $a_{j}, b_{j}, c_{j}, d_{j}$ are newly created literals.
    Let $U' = U \cup (\bigcup_{j=1}^{m} \{a_{j}, b_{j}, c_{j}, d_{j}\})$. \\

    By construction each clause of $C'$ contains exactly three literals. \\

    It is also easily verifiable that $\abs{c'_{j}} = 3\abs{c_{j}}$ since each
    clause in $C$ produces exactly 3 clauses in $C'$.  Additionally, 4 new variables
    are introduced into $U'$ for each clause in $C$, i.e. $\abs{U'} = 4\abs{U}$. \\

    Thus the transformation takes a total of $O(3|C| + 4|U|) = O(|C| + |U|)$ operations.
    I.e. the transformation can be done in polynomial time. \\

    Next we shall show that,
    \begin{align*}
      \text{C' is satisfiable if and only if C is is satisfiable}
    \end{align*}
    Let $t:U \rightarrow \{true, false\}$ be a truth assignment satisfying $C$. \\
    As $C = c_{1} \land c_{2} \land ... \land c_{2}$ is true under $t$, at least
    one literal in each $c_{j}$ must evaluate to true.
    7 cases arise: (note that unassigned variables are assumed to be false)
    \begin{enumerate}
      \item $t(x_{j}) = true$.\\
        In this case we assign $a_{j} = true$ and $b_{j} = true$.
        We thus have a single literal in each clause of $c'_{j}$ evaluated to true, and
        the expression is satisifed.
      \item $t(y_{j}) = true$. \\
        $c'_{j}$ is true, and each clause has a single true literal.
      \item $t(z_{j}) = true$ \\
        We assign $c_{j} = true$ and thus have a single positive literal in each clause,
        and the expression is satisifed.
      \item $t(x_{j}, y_{j}) = true$ \\
        We assign $a_{j} = true$ and $d_{j} = true$.
      \item $t(x_{j}, z_{j}) = true$ \\
        We assign $a_{j} = true$ and $c_{j} = true$.
      \item $t(y_{j}, z_{j}) = true$ \\
        We assign $d_{j} = true$.
      \item $t(x_{j}, y_{j}, z_{j}) = true$ \\
        See case 4.
    \end{enumerate}
    In all of the above cases,
    \begin{align*}
      \text{C is satisfiable} \Rightarrow \text{C' is satisfiable and each clause has a single true literal}
    \end{align*}

    Conversely let $t':U \rightarrow \{true, false\}$ be a truth assignment satisfying
    $C'$ such that each clause contains a single true literal. \\

    Suppose in the original 3SAT expression $\exists c_{i}, x_{i} = y_{i} = z_{i} = false$, i.e.
    $c_{i}$ is unsatisfiable.  Since $t'$ was assumed to be a valid truth assignment
    for $\Pi$, one of $b_{i}$ or $c_{i}$ must be true (otherwise we have a contradiction).
    If $b_{i}$ is true, we have $\neg x$ and $b_{i}$ both true in the same clause, which
    contradicts the assumption that $t'$ is a valid truth assignment.  Similarly
    if $c_{i}$ is true, we have $\neg z$ and $c_{i}$ belonging to the same clause,
    a contradiction of the validity of $t'$.  Therefore $c_{i}$ must be satisifiable.\\

    I.e.,
    \begin{align*}
      \text{C' is satisfiable and each clause has a single true literal} \Rightarrow \text{C is satisfiable}
    \end{align*}

    Thus we have,
    \begin{align*}
      \text{C is satisfiable} \iff \text{C' is satisfiable and each clause has a single true literal}
    \end{align*}

    Hence we have 3SAT $\alpha$ $\Pi$. \\
    Thus $\Pi \in \textbf{NP} \land \text{3SAT} \alpha \Pi$. \\
    Hence $\Pi \in \textbf{NP-}\text{Complete}$.
  \end{proof}
\end{question}

\end{document}
